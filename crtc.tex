\documentclass[a4paper,10pt]{amsart}
\usepackage{amsfonts}
\usepackage{amsmath}
\usepackage{eucal}
\usepackage{amscd}
\usepackage{url}
\usepackage[named]{algorithm}

\newcommand{\Z}{\mathbb{Z}}
\newcommand{\N}{\mathbb{N}}
\newcommand{\Q}{\mathbb{Q}}
\newcommand{\I}{\mathbb{I}}
\newcommand{\C}{\mathbb{C}}
\newcommand{\R}{\mathbb{R}}
\newcommand{\Pee}{\mathbb{P}}
\newcommand{\EuO}{\mathcal{O}}
\newcommand{\Qbar}{\overline{\mathbb{Q}}}

\newcommand{\ljk}[2]{\left(\frac{#1}{#2}\right)}
\newcommand{\modulo}[1]{\;\left(\mbox{mod}\;#1\right)}
\newcommand{\fr}{\mathfrak}

\def\notdivides{\mathrel{\kern-3pt\not\!\kern4.5pt\bigm|}}
\def\nmid{\notdivides}
\def\nsubseteq{\mathrel{\kern-3pt\not\!\kern2.5pt\subseteq}}

\newtheorem{theorem}{Theorem}[section]
\newtheorem{lemma}[theorem]{Lemma}
\newtheorem{proposition}[theorem]{Proposition}
\newtheorem{corollary}[theorem]{Corollary}
\newtheorem{definition}[theorem]{Definition}

\newenvironment{example}[1][Example]{\begin{trivlist}
\item[\hskip \labelsep {\bfseries #1}]}{\end{trivlist}}

\parindent=0pt
\parskip 4pt plus 2pt minus 2pt 

\title{Characterisation of the HD6845 CRT Controller}
\author{UtterChaos}

\begin{document}
\maketitle

\section{Introduction}

The HD6845 is commonly used as the CRT Controller (CRTC) in IBM CGA cards and
whilst its documented behaviour is well-known, there are many questions
remaining about how it operates.

Exploitation of ``glitches'' in the CRTC can lead to interesting effects used
in demos such as 8088MPH and Area5150.

Many of the internals of the 6845 can be guessed by examining the behaviour of
the chip in a cycle exact way.

We can infer that the chip consists of registers Rx and a set of internal
counters Cy which are compared with the values in registers Rx by comparators.
Actions may occur when the comparator goes high when some Cy reaches a register
value Rx and when it goes low again after reaching Rx$+1$.

We know such internal counters must exist, otherwise the values in the
registers would be lost during operation.

Of course the possibility remains that comparators compare with half or
double values or that internal counters are incremented by 2, etc.

Much of the internal construction can be inferred from block diagrams in the
datasheet, from the fact that transistors were not wasted and from the fact
the 6845 contained no circuitry for addition/subtraction, other than incrementing
of the various internal counters.

\section{Notation}

We use the terminology more or less as standardised in the blog of Reenigne.

\begin{enumerate}
\item hdot $=$ time for one pixel in hires mode (640 horizontal pixels)
\item ldot $= 2$ hdots $=$ time for one pixel in 320 pixel mode 
\item cycle $= 3$ hdots $=$ one processor cycle (aka ccycle)
\item hchar $= 8$ hdots $=$ one character in 80-column mode
\item lchar $= 16$ hdots $=$ one character in 40-column mode
\end{enumerate}

\section{Lockstep}

There are four things that can operate with a relative phase to one another:
the CPU, the Programmable Interval Timer (PIT), the CGA card, the CRTC.

All are driven from the same clock 14.318MHz clock (each cycle is 1 hdot), but
different dividers are used on the mainboard and on the CGA card.

Note that the CRTC derives its input from the CGA card, so these are by
definition in sync. Also, the CGA card timing is relevant because of the wait
states it asserts on the bus if the CPU tries to access VRAM. This is to avoid
conflict with the CGA card accessing VRAM as it continually strobes through CGA
memory for display and also VRAM refresh (which cannot be fully disabled, even
in the blanking region).

At boot time the CGA card and PIT start with a random phase with respect to
one another. As the PIT is run with a cycle of 12 hdots and the CRTC lchar is
16 hdots the best we can do is lockstep of the CRTC and PIT within 4 hdots,
meaning there is an unknown phase of 0-3 hdots between the two, which can only
be changed (randomly) on reboot.

This is relevant even if the PIT is not used as a system timer, as system DRAM
refresh is run on timer 1. DRAM refresh can be slowed or sped up from its
standard value of 18 PIT cycles (each scanline in standard modes is 76 PIT
cycles exactly), or temporarily turned off.

As the CPU clock is every 3 hdots and the 6845 lchar takes 16 hdots and these
are relatively prime, it is possible to align these.

A full CRT frame (262 scanlines of 57 lchars as defined by the IBM BIOS) is
precisely 79648 CPU cycles and exactly 19912 PIT cycles so once aligned, in
the absence of external interference, these will remain aligned in standard
modes.

This process of alignment is called \emph{lockstep} by Reenigne (a consistent
phase of the CRTC and CPU with respect to one another). Reenigne's method
of obtaining lockstep consists of a series of delays and VRAM accesses (causing
a sequence of wait states that reduces the number of possible phases at each
step).

Reenigne discovered this method by analysing the wait state algorithm of the
IBM CGA card although some experimentation was apparently required to find a
sequence that worked in practice.

Once lockstep is achieved, any fixed sequence of code, CGA register writes
(other than certain mode changes), CRTC register writes (other than those
which change the size of the CRT frame), DRAM writes and so long as there is
no reboot, PIT pulses, will result in precisely the same result down to the
hdot. If the PIT is not used (including for DRAM refresh) then the results
will be the same even after reboot.

Of course interrupts from external devices such as the keyboard, disk drives
and so on, may cause a loss of lockstep, so these are usually switched off.
The keyboard can be polled by checking the IRR and accessing the keyboard
buffer directly via a port. Lockstep can also be lost by changing the screen
geometry of course.

Unfortunately the IBM CGA card used for the experiments in this document
did not respond correctly to Reenigne's lockstep sequence (we have not yet
determined the reason). Therefore we use a slightly different approach.

Similarly to the last part of Reenigne's method, we set up a CRTC frame of
four horizontal two characters by two scanlines in CGA graphics mode so that
there are 8 lchars per frame. We set horizontal displayed and vertical
displayed to 1 so that there is exactly one lchar in the active region.

We first use Reenigne's loop of exactly 144 cycles which is exactly 27
lchars to step through lchars until we reach the active lchar. At this point
there are exactly 16 phases (one for each hdot of the lchar). We then waste
some cycles until we are just back into the lchar immediately following the
active lchar. We then use a loop of precisely 85 CPU cycles which is 255
hdots, or one hdot less than a multiple of 4 lchars. This allows us to step
back to the active lchar one hdot at a time until it is again detected. At
this point we are in full lockstep.

The lockstep algorithm we use appears to be reliable. The display enable
signal appears to be consistent down to the hdot, though obviously we don't
know about any delays due to internal circuitry. By the time the CPU knows
about the display enable signal, some time may have passed. But at least this
delay is absolutely consistent, to the hdot (modulo hardware defect).

Note that there are three possible hdot phases of the first hdot of the
active region in a standard mode with respect to the CPU cycle that occurs at
that time (0, 1 or 2 hdots).

To precisely control the phase we again step a fixed number of times by 85
cycles to get a phase of 0, 1 or 2 hdots as desired.

\section{Limitations}

In order to determine what time various register writes take effect, and the
precise time hdot marginal behaviour is exhibited, there are three methods that
could be used.

\begin{itemize}
\item Assume the CRTC is a black box and analyse any delays in the CGA card
and system board schematics
\item Assume the CRTC is a black box and instrument the system with a sniffer
\item Determine timings from their visual effect
\end{itemize}

In this document we have decided to take the third approach. However, in order
to do so, there must be some reference event that is visible that we can
compare everything to.

As CGA background register writes take effect down to the hdot (visible using
an RGB2HDMI adaptor with a large LCD monitor) and probably take effect with
very little delay, we decided to give all timings with respect to when a
background register change causes a visible change onscreen.

To this end, the precise hdot where the colour change becomes visible for the
first time is taken to be the time that the background colour register change
took effect. Of course this subsumes a constant delay due to internal CGA
circuitry which is unknown.

To write a CRTC register takes one byte to be sent to port 3D4H to specify the
CRTC register and another byte to be sent to port 3D5H to specify the value.
Obviously the register value cannot change until the second byte is received,
and again we don't know about any internal delays.

It's conceivable that the delay depends on which bits are changed or on whether
the value is being changed \emph{to} a value that corresponds to a character
position about to be encountered imminently or \emph{from} such a value. We
refer to the latter as a \emph{to-imminent} or \emph{from-imminent} changes,
respectively.

Each write of a byte to a port takes approximately 4-5 lchars including the time
to load the port address and value into CPU registers (the precise number of
cycles depends on at least the state of the CPU prefetch, bus activity and
whether it has been idle, interrupts and possibly other things). Thus a write of
both bytes to change a CRTC register is a relatively slow process.

In light of unknown constant delays associated with both background register
writes and CRTC register writes we cannot talk in absolute terms about when
these took place. However, we can talk about when one occurs relative to the
other.

Obviously the relative delays will depend on model of video card, but the
timings we give in this document are for one of the two IBM CGA cards which
behave identical up to differences in the CRTC chip.

In order to measure the relative timings of various effects that occur on a
CRTC register write we first do \emph{two} background register changes (each of
which can be done with a write of a single byte to a CGA port), e.g. we change
the background colour to red on the first write and then back to black on the
second write. We then use exactly the same code, but change the two background
register writes to a single CRTC register write (recall this takes two bytes to
be written to ports).

There are no wait states associated with any of these writes on the IBM PC,
therefore from the CPU's perspective the code will take precisely the same
number of cycles in both cases when this exchange is made in the code.

Obviously we are interested in when the CRTC register change takes effect,
and we \emph{define} it to have been written when the second background colour
change was visible onscreen (which we will have previously determined before
the code was switched).

What we are interested in is when a CRTC register value has to be written (as
measured in the way just specified) in order for it to take effect, or to
exhibit some glitch or other marginal behaviour.

In this way we aren't really measuring when such register writes take effect
because of the unknown internal delays, but we have a relative reference point
with which to compare.

This method can't tell us anything about any constant internal delays, but it
can conceivably tell us about any delays that depend on which bits are changed
or on whether a to-imminent or from-imminent change is being made.

Obviously if instrumentation or circuit analysis or simulation subsequently
reveal something about absolute delays associated with a background register
write then we can use the relative information recorded in this document to
say something in absolute terms about when the CRTC register writes actually
take effect. However, from the programmer's perspective this information isn't
needed to be useful.

\section{The Horizontal Displayed Register}

This controls the number of characters that are displayed horizontally on each
scanline, but many questions remain.

\begin{enumerate}
\item Precisely when does it take effect?
\item Is there any edge case behaviour?
\item What resets the count to zero?
\item What triggers when the comparator goes high/low?
\item What happens if it is set multiple times in a scanline?
\item What is its behaviour near other register values, such as hsync pos,
      and htotal?
\item When is the horiz. disp. value added to the address?
\end{enumerate}

\subsection{Methodology}

The program HDISP1.ASM does the following to achieve lockstep:

\begin{enumerate}
\item Enter CGA graphics mode (equivalent of BIOS mode 4)
\item Set up a CRTC frame with 2 scanlines of 4 lchars, with one lchar in the active region
\item With a cycle exact loop of 144 cycles $= 4n-3$ lchars ($n = 7$) wait until in the active region
\item Waste some cycles until back in the inactive region
\item With a cycle exact loop of 85 cycles, wait until in the active region again
\end{enumerate}

This puts the CPU in a known state with respect to the CRTC down to the hdot.
As far as we can tell, this is entirely consistent between runs if DRAM refresh
is disabled.

In order to handle DRAM refresh the program executes 256 consecutive bytes of
code in each 64kb block of system RAM, repeated four times per CRT frame. This
causes those 256 bytes to be read approximately every 4ms.

The 256 bytes per 64kb block are alternated between offset 0 and offset 256.
The reason for this is that on the XT 4164 and 41256 DRAM chips are typically
used. The 4164's require refresh every 4ms, or roughly four times per CRT
frame. The 4164's are 64kbits and the 41256's are 256kb. The former will be
refreshed if reads are done on 256 rows (corresponding to the low 8 address
bits) and the latter require reads on 512 rows (corresponding to the low 9
address bits).

The alternating behaviour of the reads from offset 0 to offset 256 ensures
that even for the 41256 chips, all 512 rows are energised.

To execute code in all 64kb blocks, small segments of 256 bytes in each block
are saved by our program and overwritten by NOPs and a RETF. These are replaced
before returning to DOS.

Note that this method does not work on early PCs, as these used 16kbit 4116
chips which need refreshing every 2ms. As there are so many of them in a 640kb
system, there isn't time in a CRT frame to refresh them all.

The program turns off the keyboard and timer interrupts (the only ones firing
on a quiet XT) and polls the IRR to see if a keyboard IRQ has occurred. If this
is detected, the entire program restarts, usually modifying some program values
before doing so, in response to the keypress. Lockstep is re-run on every
restart to ensure consistent behaviour between runs.

The main loop can be adjusted to a precise number of cycles and defaults to the
exact number of cycles for a standard CGA CRTC frame.

The code for changing background colour/CRTC registers and for polling the
keyboard is interleaved with the refresh code so that there is no interference
between DRAM refresh and these other operations.

The register writes always happen at the top of the main loop, however their
exact position relative to the CRTC frame down to the hdot can be changed
by altering a delay after lockstep and before the start of the main loop to
adjust the cycle the loop starts on, and an additional delay of some multiple
of 85 cycles after lockstep but before a normal CRTC frame size is
reestablished in order to control the hdot phase of 0-2 hdots.

Longrunning instructions such as SHL (with a high count) and MUL are used for
timing within 4 and 1 cycles respectively.

The program can be made to switch from background register writes to a
horizontal displayed (hdisp) register write and back again. When doing
background register writes the program displays pixels on every second
scanline, with alternating on/off pixels, starting with the first on each
scanline. This allows the background colour changes to be observed when
measured against the coloured pixels in the neighbouring scanlines.

Different coloured pixels are used in each lchar in a repeating pattern.

When doing CRTC register writes all pixels on the screen are filled (again
repeating a pattern to delineate the lchars) except for every tenth scanline
which is left black. The latter is done to make it easier to see when the
CRTC has stopped updating the VRAM address at the end of each scanline.

Counting is done on a large LCD monitor after conversion using an RGB2HDMI
adaptor, which makes every hdot visible.

\subsection{When does the horiz. disp. reg. value take effect?}

For a to-imminent change from a higher hdisp value we see three behaviours:

\begin{enumerate}
\item The new hdisp value ($n$) takes full effect, blanking all but the
first $n$ characters
\item An intermediate behaviour where the new hdisp value only takes effect
from the second half of the lchar where it is supposed to start blanking, leaving
an extra hchar displayed before blanking the remainder of the scanline
\item The hdisp value does not take effect, causing all characters to be
displayed right out to htotal$+1$ characters, with the exception
of the final character which only displays the first hchar of the full lchar,
the right hand hchar being blank
\end{enumerate}

These behaviours occur when the write happens on the fourth last, third last or
second last hdot before the imminent character, respectively.

A to-imminent change to a strictly greater value has no visible effect on the
current scanline, as we are already blanking at that point, so it continues to
do so. However, there is a revealing behaviour.

So long as the higher value is written to the register strictly before the
second last hdot of the higher hdisp is reached (and we are on the last
scanline of the character row), the address will be updated to the higher value,
even though characters have not been displayed since the lower hdisp value was
reached.

This behaviour occurs so long as the higher value is written to the register
strictly after the fifth hdot before the lower value is reached.

A from-imminent change to a lower hdisp value causes the third behaviour in the
list above if the change is made early enough (strictly before the 8th hdot of
the character). As the hdisp value is never reached in this case the address is
not updated.

As might be expected, a from-imminent change to a higher hdisp value results in
no unusual behaviour (other than the address incrementing behaviour just
described). If the change is made in time the larger value for hdisp is used,
otherwise the smaller value continues to be used.

However, the cutoff between the two cases changes according to a very
intriguing rule which we have verified in at least 3 HD6845 chips (and not in
other CRTC chips).

Generally, the larger value must be written strictly before the 8th hdot of the
character before the smaller value is reached. However in very specific cases
there is an extra one hdot delay meaning that the write must be done one hdot
earlier to take effect.

If we are making a from-imminent change of hdisp from $m$ to some $n > m$ then
the extra hdot delay exists if all of the bits of $m$ which are set are also
set in $n$, with the exception of bit $0$ which does not affect timings.

Note that it doesn't make any difference if additional bits of $n$ are set. The
only ones that affect the timing are those corresponding to bits set in $m$
(with the exception of bit $0$ of course).

We have not been able to speculate what causes this unusual timing behaviour,
but it has been very carefully measured.

\subsection{Is there any edge case behaviour?}

If hdisp is set strictly greater than htotal the final character of the
scanline is only half displayed; the second hchar of that final lchar is blank.

This is probably a general behaviour of this CRTC (the final hchar before
the end of the scanline is blanked).

\subsection{What resets the count to zero?}

Unknown, however a single internal counter is likely used for comparison
against the hdisp, htotal and hsync pos registers, and this horizontal counter
is needed for comparison against htotal, so we know the counter can't be reset
before htotal is reached. It must be 0 by the time htotal$+1$ is reached as
this is the start of the next scanline where hdisp may be $0$, so the counter
must be $0$ by that point.

\subsection{What triggers when the comparator goes high/low?}

Address update seems to be triggered by the hdisp comparator so long as hdisp
is reached before the horizontal counter reaches htotal$+1$ characters (see
below).

Interestingly, if hdisp is not reached on the final scanline of the frame
the address is not reset to the start address but continues to increment in the
next frame.

This is slightly surprising as one might have imagined that this had to happen
at the end of the scanline on which the horizontal total adjust value is
reached.

The exact sequence of events that trigger address reset is not clear, but it
seems likely that hdisp being reached is a component of it, just as it is for
address update at other times.

It's unlikely that it's simply the vertical total adjust counter reaching
the vertical total adjust value as this value could be zero and the counter is
likely only 5 bits meaning that it isn't counting all the time and is therefore
likely zero at other times when hdisp is reached.

\subsection{What happens if it is set multiple times in a scanline?}

The program HDISP2.ASM allows two updates to hdisp to be made on the same
scanline. Both values can be set independently.

All behaviour appears to be in line with the following:

\begin{itemize}
\item Whenever hdisp is reached on the last scanline of a row, the address is
      updated.
\item Reaching hdisp the first time begins blanking and this continues to the
      end on the scanline regardless if hdisp is reached again or not
\item If hdisp is not reached for a first time the entire scanline is filled
      (with the exception of the last hchar as usual), but the address is not
      updated
\end{itemize}

\subsection{What is its behaviour near other register values, such as hsync pos,
and htotal?}

See the other sections for behaviour near htotal.

There appears to be no special behaviour when changing hdisp near hsync pos.

\subsection{When is the hdisp value added to the address?}

If the hdisp value is strictly greater than htotal the address is not updated,
otherwise it is updated on the last scanline of a row when hdisp is reached.

\section{The Horizontal Total Register}

This controls the number of characters horizontally in the scanline. The actual
number of characters is always one more than the value in the register. One
can speculate that this is because there are tasks that the CRTC needs to do in
the final character of each scanline and so the comparator needs to go high one
character before the end of the scanline.

There are a number of questions we'd like to answer about the horizontal total
register.

\begin{enumerate}
\item When does htotal take effect?
\item Is there any anomalous behaviour?
\item What is triggered by the comparator going high?
\end{enumerate}

\subsection{Methodology}

There are only so many values that we can easily set the horizontal total to.
Typically a CRTC frame is 57 lchars wide. Monitors tend to tolerate values
very close to this.

One consideration is that if the number of characters in a scanline isn't
divisible by 3 there will not be an whole number of CPU cycles in a frame
unless the number of scanlines is divisible by 3. We can ensure this is the
case by adjusting the vertical total adjust.

The program HTOT1.ASM sets the width of the screen to 56 lchars and for three
rasterlines sets the size of a scanline to 28 lchars. In order to compensate
for three extra scanlines being used up in this region, we have to add an
additional three scanlines to the vertical total adjust, as well as round it
to a multiple of 3.

The program HTOT2.ASM sets the width of the screen to 57 lchars and for three
rasterlines sets the size of a scanline to 19 lchars. Once again we have to
compensate for the extra scanlines being used up in this region by adding
6 in this case to the vertical total adjust.

\subsection{When does htotal take effect?}

Without exception it seems that htotal can be written up to 18 hdots before the
scanline we want to have length htotal$+1$. However it does not need to be set
before the scanline starts. So long as it is set strictly before 18 hdots ahead
of the next scanline (both the existing one and the one we are setting up), it
will still take effect.

Setting htotal after this point will cause a wraparound of the horizontal
counter (which counts to 255 before wrapping around to zero) and for the
counter to keep going until it encounters the new htotal value after the
wraparound. This very likely results in loss of sync.

\subsection{Is there any anomalous behaviour?}

If one sets htotal to a smaller value than hdisp for the last scanline of the
frame, the address will not be reset to the start address at the beginning of
the next frame. This causes the address to keep incrementing through VRAM even
as we go from frame to frame.

\subsection{What is triggered by the comparator going high?}

Presumably the horizontal counter is reset to zero ready for the next scanline.

Likewise it seems probable that the row address is incremented.

It's unknown whether the horizontal sync counter could be reset at this time.

Precisely when these events happen hasn't been determined, and we don't know
if they occur when the comparator for htotal goes high or low.

\end{document}
