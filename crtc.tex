\documentclass[a4paper,10pt]{amsart}
\usepackage{amsfonts}
\usepackage{amsmath}
\usepackage{eucal}
\usepackage{amscd}
\usepackage{url}
\usepackage[named]{algorithm}

\newcommand{\Z}{\mathbb{Z}}
\newcommand{\N}{\mathbb{N}}
\newcommand{\Q}{\mathbb{Q}}
\newcommand{\I}{\mathbb{I}}
\newcommand{\C}{\mathbb{C}}
\newcommand{\R}{\mathbb{R}}
\newcommand{\Pee}{\mathbb{P}}
\newcommand{\EuO}{\mathcal{O}}
\newcommand{\Qbar}{\overline{\mathbb{Q}}}

\newcommand{\ljk}[2]{\left(\frac{#1}{#2}\right)}
\newcommand{\modulo}[1]{\;\left(\mbox{mod}\;#1\right)}
\newcommand{\fr}{\mathfrak}

\def\notdivides{\mathrel{\kern-3pt\not\!\kern4.5pt\bigm|}}
\def\nmid{\notdivides}
\def\nsubseteq{\mathrel{\kern-3pt\not\!\kern2.5pt\subseteq}}

\newtheorem{theorem}{Theorem}[section]
\newtheorem{lemma}[theorem]{Lemma}
\newtheorem{proposition}[theorem]{Proposition}
\newtheorem{corollary}[theorem]{Corollary}
\newtheorem{definition}[theorem]{Definition}

\newenvironment{example}[1][Example]{\begin{trivlist}
\item[\hskip \labelsep {\bfseries #1}]}{\end{trivlist}}

\parindent=0pt
\parskip 4pt plus 2pt minus 2pt 

\title{Characterisation of the HD6845 CRT Controller}
\author{UtterChaos}

\begin{document}
\maketitle

\section{Introduction}

The HD6845 is commonly used as the CRT Controller (CRTC) in IBM CGA cards and
whilst its documented behaviour is well-known, there are many questions
remaining about how it operates.

Exploitation of ``glitches'' in the CRTC can lead to interesting effect used
in demos such as 8088MPH and Area 5150.

Many of the internals of the 6845 can be guessed by examining the behaviour of
the chip in a cycle exact way.

We can infer that registers Rx have a corresponding internal counter Cx which
is compared with the value in Rx by some comparator. Actions may occur when
the comparator goes high and when it goes low.

We know such internal counters must exist, otherwise the values in the
registers would be lost during operation.

Of course the possibility remains that comparators compare with half or
double values or that internal counters are incremented by 2, etc.

\section{Notation}

We use the terminology as standardised in the blog of Reenigne.

\begin{enumerate}
\item hdot $=$ time for one pixel in hires mode (640 horizontal pixels)
\item ldot $= 2$ hdots $=$ time for one pixel in 320 pixel mode 
\item cycle $= 3$ hdots $=$ one processor cycle (aka ccycle)
\item hchar $= 8$ hdots $=$ one character in 80-column mode
\item lchar $= 16$ hdots $=$ one character in 40-column mode
\end{enumerate}

\section{Lockstep limitations}

There are four things that can operate with a relative phase to one another:
the CPU, the Programmable Interval Timer (PIT), the CGA card, the CRTC.

All are driven from the same clock 14.318MHz clock (each cycle is 1 hdot), but
different dividers are used on the mainboard and on the CGA card.

Note that the CRTC derives its input from the CGA card, so these are by
definition in sync. Also, the CGA card timing is relevant because of the wait
states it asserts on the bus if the CPU tries to access VRAM. This is to avoid
conflict with the CGA card accessing VRAM as it continually strobes through CGA
memory for display and also DRAM refresh (which cannot be disabled, even in the
blanking region).

At boot time the CGA card and PIT start with a random phase with respect to
one another. As the PIT is run with a cycle of 12 hdots and the CRTC hchar is
8 hdots the best we can do is lockstep of the CRTC and PIT within 4 hdots,
meaning there is an unknown phase of 0-3 hdots between the two, which can only
be changed on reboot.

This is relevant even if the PIT is not used as a system timer, as system DRAM
refresh is run on timer 1. DRAM refresh can be slowed or sped up from its
standard value of 18 PIT cycles (each scanline in standard modes 76 PIT
cycles exactly), or temporarily turned off.

As the CPU clock is every 3 hdots and the 6845 hchar takes 8 hdots and these
are relatively prime, it is possible to align these. In practice this is done
by a series of delays and VRAM accesses and then waiting for a particular
lchar in a CRTC frame. This puts the CRTC and CPU into lockstep (a consistent
phase with respect to one another, though at any given time one doesn't know
what this phase actually is).

A full CRT frame (262 scanlines of 57 lchars) is a precise number of CPU cycles
and exactly 19912 PIT cycles. Of course most monitors have some tolerance and
we are referring here to the CRT frame as set up in IBM BIOS modes.

Reenigne has developed code for lockstep to the maximum possible degree modulo
the unalterable phase between the CGA card and CRTC.

Subject to the system timer not being used and DRAM refresh being off or only
occuring when there is no other bus activity, this lockstep can be made cycle
accurate.

We also make use of a partial lockstep in this document which consists of the
second part of Reenigne's lockstep, followed by an additional section to bring
things within one CPU cycle of lockstep, meaning that there is a consistent
phase of 0-2 hdots which is not determined.

\section{The Horizontal Displayed Register}

This controls the number of characters that are displayed horizontally on each
scanline, but many questions remain.

\begin{enumerate}
\item Precisely when does it take effect?
\item Is there any edge case behaviour?
\item What resets the count to zero?
\item What triggers when the comparator goes high/low?
\item What happens if it is set multiple times in a scanline?
\item What is its behaviour near other register values, such as horiz. sync.pos,
      and horiz. total.
\item When is the horiz. disp. value added to the address?
\end{enumerate}

\subsection{Methodology}

The program HORDISP.ASM does the following to get a partial lockstep:

\begin{enumerate}
\item Enter CGA graphics mode (equivalent of BIOS mode 4)
\item Set up a CRTC frame with 2 scanlines of 2 lchars, with one lchar in the active region
\item With a cycle exact loop of 144 cycles $= 4n-3$ lchars ($n = 7$) wait until in the active region
\item Waste some cycles until back in the inactive region
\item With a cycle exact loop of $4k$ lchars minus 1 cycle, wait until in the active region again
\end{enumerate}

This puts the CPU in a known state with respect to the CRTC, within 1 cycle (3
hdots) every time.

The program turns off the keyboard and timer interrupts (the only ones firing
on a quiet XT) and polls the IRR to see if a keyboard IRQ has occurred. If this
is detected, the entire program restarts, usually modifying some program values
before doing so, in response to the keypress.

The main loop can be adjusted to the precise number of cycles for a standard
CGA CRT frame and DRAM refresh is set to 38 PIT cycles to localise interference
to the same two positions on each scanline.

The keyboard polling is also done at a consistent time about half way down the
frame and its position can be adjusted in a cycle accurate way.

Longrunning instructions such as SHL (with a high count) and MUL are used for
timing within 4 and 1 cycles respectively. These instructions have no bus
access associated with them, except for instruction prefetch. The code is
written so that as many instruction prefetches as practicable occur at the same
relative place on a scanline to minimise interference from system DRAM refresh.

The program allows setting the horiz. disp. reg. at any time to within about 1
cycle. Often simply restarting by pressing a key causes a different phase (0-2
hdots) at the point the register is being written, due to the imperfect
lockstep, allowing easy exploration of all three behaviours.

The program has two modes, one which sets the background register of the CGA
to put a short piece of red background on a scanline at two places on the
screen. The first pixel after this red mark is taken to be the point at which
this register write took place.

This can then be swapped for code that writes a CRTC register instead (the
horiz. disp. reg.) at precisely the same point (by using the same code just
with different values for the register port address and value).

This allows one to visually control almost down to the pixel (really to
within 1 CPU cycle) where the CRTC register write happens.

Once approximate information is gained in this way, we may write a program that
uses full lockstep and precise timing to get information down to the hdot.

\subsection{When does the horiz. disp. reg. value take effect?}

We do not know precisely when the background colour change occurs relative
to when the register is written, and there is a similar ambiguity for the
CRTC register write. Therefore all we can really report without instrumentation
is when the CRTC register write \emph{appears} to take place as determined by
the visual change that a similar background register write causes when done at
precisely the same time in the loop (modulo the cycle uncertainty - the program
restarts when switching from a background register write to a CRTC register
write).

We see three behaviours which can be described as follows:

\begin{enumerate}
\item The horiz. disp. value ($n$) takes full effect, blanking all but the
first $n$ characters
\item The horiz. disp. value does not take effect, causing all characters
to be displayed right out to horiz. total $+1$ characters, with the exception
of the final character which only displays the first hchar of the full lchar,
the right hand hchar being blank
\item An intermediate behaviour where the horiz. disp. value only takes effect
from the second half of the lchar where it is supposed to start blanking, leaving
an extra hchar displayed
\end{enumerate}

As far as can be determined using HORDISP.ASM, the dividing line between the
three behaviours occurs when the register write occurs on the second last pixel
of character $n - 1$ (numbered from $0$) where $n$ is the value being written
into the horiz. disp. reg., i.e. the second last pixel before it is supposed
to take effect.

The precise hdot and for how many hdots (1 or 2) that the intermediate
behaviour occurs is only known approximately.

Given that all three behaviours can be exibited within the same cycle it is
likely that there is only one hdot exhibiting the intermediate behaviour
and experiments suggest this could be the third last hdot of the lchar. This
however remains to be confirmed.

\subsection{Is there any edge case behaviour?}

If horiz. disp. is set greater than or equal to horiz. total $+1$ the final
character of the scanline is only half displayed; the second hchar of that
final lchar is blank.

This is probably a general behaviour of this CRTC (the final hchar before
the end of the scanline is blanked).

\subsection{What resets the count to zero?}

Unknown.

\subsection{What triggers when the comparator goes high/low?}

Unknown.

\subsection{What happens if it is set multiple times in a scanline?}

Unknown.

\subsection{What is its behaviour near other register values, such as horiz. sync.pos,
      and horiz. total.}

Unknown.

\subsection{When is the horiz. disp. value added to the address?}

Unknown.

\end{document}
